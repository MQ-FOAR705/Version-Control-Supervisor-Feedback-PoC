\documentclass{article}
\usepackage{geometry}
\usepackage[utf8]{inputenc}
\usepackage{graphicx}
\graphicspath{ {images/} }
\geometry{a4paper, left=20mm, right=20mm, top=25mm, bottom=25mm}

\title{Version-Control-Supervisor-Feedback-PoC}
\author{Jeremy Amin}
\date{November 2019}

%\pagestyle{plain}
\usepackage{fancyhdr}
\pagestyle{fancy}
\fancyhf{}
\rhead{J. Amin - 42107881}
\lhead{FOAR705 - Supervisor Feedback PoC}
\lfoot{Session 2, 2019 - Macquarie University}
\rfoot{\thepage}

\usepackage[colorlinks=true,linkcolor=blue]{hyperref}
\usepackage{hyperref}

\begin{document}

\maketitle
\tableofcontents

\pagebreak

\section{Mary is hungry}

Consider a situation where Mary is hungry. In this situation, it is good for Mary to eat an apple and Mary has a right to acquire an apple to eat. It is considered bad for Mary to eat and swallow shards of glass in this scenario and it would be a duty other people have to not feed her shards of glass, and Mary would have some kind of authority to impose onto others her right to not be coerced into being force fed glass. In this scenario, the apple is valuable to Mary and the glass is not valuable. We would say that the value of the apple or the glass is relative to some feature of the scenario which is more fundamental, in this case the life and health of Mary herself. Mary is more valuable than the food which she consumes, and this is because Mary as the being who is doing the eating is that which we make the judgement of better and worse, and good and bad, about. From this analysis, the health and existence of Mary are more valuable than the food which she eats. The food is merely an instrumental good consumed for the sake of something more valuable than the food itself, namely Mary herself.

\section{Ethics on A-T}

On Aristotelian-Thomistic philosophy, ethics is the area of study where human beings are the subjects of analysis. The concepts of good and bad, right and wrong, etc. begin to take on a more specific definition when it comes to human life because we have the ability to understand reasons, and are free to act or not act based on reasons. The will and our rational minds transform the way we relate to what is good for us because we have the ability to choose to act in such a way that is in harmony with the goods that fulfil our natures, and we can also act in such a way as to stifle our natures by pursuing ‘‘goods’ which are not in fact what perfect our natures.

\end{document}

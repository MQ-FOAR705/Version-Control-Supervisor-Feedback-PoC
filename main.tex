\documentclass{article}
\usepackage{geometry}
\usepackage[utf8]{inputenc}
\usepackage{graphicx}
\graphicspath{ {images/} }
\geometry{a4paper, left=20mm, right=20mm, top=25mm, bottom=25mm}

\title{Version-Control-Supervisor-Feedback-PoC}
\author{Jeremy Amin}
\date{November 2019}

%\pagestyle{plain}
\usepackage{fancyhdr}
\pagestyle{fancy}
\fancyhf{}
\rhead{J. Amin - 42107881}
\lhead{FOAR705 - Supervisor Feedback PoC}
\lfoot{Session 2, 2019 - Macquarie University}
\rfoot{\thepage}

\usepackage[colorlinks=true,linkcolor=blue]{hyperref}
\usepackage{hyperref}

\begin{document}

\maketitle
\tableofcontents

\pagebreak

\section{Introduction}

This document is my new PoC. My plan is to create a workflow which allows me to receive supervisor feedback in a version controlled location, implement said feedback, and then send a report to my supervisor demonstrating that I have implemented their feedback.

\section{Scoping I}

\subsection{A Day in the Life: Feedback from my supervisor on the written component of my thesis}

I need to:
\begin{itemize}
    \item Have the meeting with my supervisor and record their feedback on my progress.
    \item Make sure I record important actionable feedback in such a way that it is easily understood by myself and them at a later date.
    \item Save the feedback in a version controlled format which I will not lose even if my computer crashes.
    \item Implement their feedback and record it precisely so that I can demonstrate that I have met their demands/suggestions.
    \item Send the implemented actions to my supervisor in an easy to use format for them. (Probably Txt, Docx, or PDF)
    \item Just in case there is a contradiction between their initial feedback and my appropriate implementation of said feedback, I will have a copy of what they have suggested ready to hand.
    \item Just in case they wish to see precisely what I have changed, I will have a history of my edits to my thesis so I can show what I changed and how it changed.
\end{itemize}

\subsection{\textit {Pains} and pain relievers}

\begin{itemize}
    \item \textit{Pain 1: Manually searching through my document(s) to find the specific alteration(s) I have made.} 
    \begin{itemize}
        \item  Reliever: Having a version controlled system in place which enables me to have a history of my alterations as well as an easy way to compare older versions of what I have written to compare with the updated text based on supervisor feedback. 
    \end{itemize}
    \item \textit{Pain 2: Demonstrating precisely why my alteration is better than the previous version.} 
    \begin{itemize}
        \item  Reliever 1: A set of themed repositories and folders which enable me to search for my feedback based alterations based on: Date of feedback, time and date feedback was implemented, section of thesis, page number.
        \begin{itemize}
        \item Reliever 1.1: Being able to compare the current version of a section of text with its older versions. I would then send the relevant versions to my supervisor so they can see the precise alterations I made and see how the alterations satisfy their feedback.
        \end{itemize}
    \end{itemize}
\end{itemize}

\subsection{\textit {Gains} and gain creators}

\section{Scoping II: Computational Analysis}

\subsection{Instructions}
Using elements of ‘computational thinking’, consider the pains, gains, and solutions you identified during BA. Translate them into a form that can (at least in part) be addressed by computers. To do so, consider these steps:
\begin{itemize}
    \item Decomposition:Breaking down data, processes, or problems into smaller, manageable parts
    \item Pattern Recognition:Observing patterns, trends, and regularities in data.
    \item Algorithm Design:Developing the step by step instructions for solving this and similar problems
\end{itemize}

\subsection{Goal}

To create a workflow, preferably with some kind of code, to eliminate at least one manual step in the chain of action from receiving feedback to demonstrating that I have implemented feedback.

\subsection{Decomposition}

The process of receiving feedback from my supervisor may proceed as follows:

\begin{itemize}
    \item Commit Overleaf project to GitHub
    \item Download TeX file from Overleaf onto my computer
    \item Convert TeX file to Docx
    \item Send Docx file to supervisor
    \item Receive commented Docx file from supervisor
    \begin{itemize}
    \item Download the Docx file and open it in Microsoft Office.
    \end{itemize}
    \item Find the sections with comments in the document.
    \item Copy the feedback from the comment boxes to Overleaf.
    \item Make alterations to the Overleaf file.
    \item Once all alterations are done to my satisfaction, commit the changes to GitHub and write a commit message which notes the changes made.
    
\end{itemize}

\subsection{Pattern Recognition}

The patterns I am going to focus on are:

\begin{itemize}
    \item

\end{itemize}

\subsection{Algorithm Design}

The algorithm is made up of:

\begin{itemize}
    \item

\end{itemize}


\section{Elaboration I}

\subsection{Task Outline}

\noindent
In Scoping I identified two patterns I wish to make easier. There were:

\begin{itemize}
    \item Finding specific words and sentences within and across different papers in an efficient manner
    \item Collecting and organizing papers into a single bibliographic database in such a way that I can search for papers based on a theme.

\end{itemize}

\noindent
Depending on the software I can find, I may modify the second bullet point to make it a simpler goal for my PoC. I will also include a referencing goal as a part of my PoC.

\begin{itemize}
    \item Find software which will serve as a reference database and which will do the referencing work for me 
\end{itemize}


\noindent
I notice that I did not make the Algorithm part of Scoping II detailed enough. It seems I misunderstood the difference between it and the Decomposition section. In the Elaboration process section I rectify this mistake.


\subsection{Potential Software to solve my problems}

\subsubsection{Terminal on OSx}

Downloading TeX script from Overleaf git clone.

\begin{itemize}
    \item Command to download a file from the internet: curl -O URL of file
    \item 
\end{itemize}

\subsubsection{OverLeaf}

Document writing with more control over the formatting etc. compared to Word or Google Documents. Also enables version control in a seamless way.

\subsubsection{GitHub}

This will be where I have the online version controlled repository for the feedback.

\subsubsection{Pandoc}

Pandoc will allow me to convert from TeX to Docx. I may need to convert TeX to Markdown, then Markdown to Docx

\subsubsection{pdf2docx.com/}

This is a free website which converts PDF's to Docx

\subsection{Elaboration process}

The steps I need to take to test the software listed above are as follows:

\begin{itemize}
    \item 
\end{itemize}

\section{Elaboration II}

\section{Proof of Concept Design}

\subsection{Instructions}

\subsubsection{Workflow steps}

\begin{enumerate}
    \item Download and install Git onto your computer. Go to this website to find the download links `https://git-scm.com/downloads'
    \item Download and install Pandoc onto your computer. Go to this website to find the download links `https://pandoc.org/installing.html'
    \item 
    \item Open Terminal
    \item In Terminal, figure out current directory. Type `ls' into the command line in Terminal.
    \item Change directories so that you are where you want to be when downloading the TeX project. My target directory, named Project, is on my Desktop, so I typed into the command line `cd /Users/JeremyAmin/Desktop/Project'
    \item Next step is to download the project from Overleaf. To do this: 
    \begin{enumerate}
        \item Go to the website and when in a project click on the `Menu' button.
        \item Under the `Sync' heading, click on the `Git' button.
        \item Copy the entire link `git clone https://git.overleaf.com/[yourprojectnumber]'
        \end{enumerate}
    \item Go back to Terminal and paste the link into the desired directory.
\end{enumerate}

\section{Disaster Recovery Plan}

\section{Learning Journal for new PoC}

\subsection{7 NOV}

1:30 PM
\begin{itemize}
    \item Intention: Figure out how to convert TeX document to Docx.
    \item Action: Spoke with Brian in a consultation. He send me this article to read: https://medium.com/@zhelin\-chen91/how-to-convert-from-latex-to-ms-word-with-pandoc-f2045a762293
    \item Outcome: It seems that Pandoc is a very good candidate software for converting TeX into Docx. Pandoc is both free and open source, two positives for a piece of software I am intending on using in a workflow for multiple users.
\end{itemize}

1:46 PM
\begin{itemize}
    \item Intention: Download Pandoc for my Macbook
    \item Action: Went to: https://pandoc.org/installing.html to find the installer for Mac. Clicked on the button at the top of the page `Download the latest installer for macOS.' Followed instructions and installed Pandoc. 
    \item Outcome: It seems that I need to have either an older version of Pandoc, or I need to install a newer macOS. On the Homebrew Formulae website it states in the `bottle' section `mojave, high\textunderscore sierra, sierra'. So it seems the current version of Pandoc is only compatible with those three macOS versions.
    \item Solution: I will look for an older version of Pandoc compatible with OS X El Capitan Version 10.11.6. If I cannot find a compatible version of Pandoc I may have to update my OS. I have found a number of programs do not run on my OS. Considering my OS is a number of years old I am not surprised.
\end{itemize}

2:05 PM

\begin{itemize}
    \item Intention: Find a version of Pandoc which works on my Macbook
    \item Action: Went to: https://stackoverflow.com/questions/48430440/install-older-version-of-pandoc-2-using-homebrew from a google search for compatible versions of Pandoc with macOS El Capitan 10.11.6.
    \item Outcome: The answer provided three solutions. One option was to use old formulas. The second was to use old bottles. The third was to ``Build your own retro-formula". I decided to try the second option.
    \item Following the instructions I went to this website: https://bintray.com/homebrew/bottles/pandoc/1.16.0.\-2\#files which had this file https://bintray.com/homebrew/bottles/download\textunderscore  file?file\textunderscore path=pandoc-1.16.0\-.2.el\textunderscore capitan.bottle.tar.gz
    \item Solution: I will look for an older version of Pandoc compatible with OS X El Capitan Version 10.11.6. If I cannot find a compatible version of Pandoc I may have to update my OS. I have found a number of programs do not run on my OS. Considering my OS is a number of years old I am not surprised.
\end{itemize}

2:35 PM

\begin{itemize}
    \item Intention: Figure out how to remove hbox warning in Overleaf when I paste website URL's into the document.
    \item Action: Went to TeX Stack Exchange to find solution.
    \item Outcome: Found solution. I decided to hypenate where appropriate to make the URL fit into the document.

    \item Intention: Figure out how to add \# character into document.
    \item Action: Went to TeX Stack Exchange to find solution.
    \item Outcome: Found solution. Adding {\textbackslash} before the \# solved my problem.
    \item Note: Each of these characters have special meaning in TeX. \& \% \$ \# \_ \{ \} \textasciitilde \textasciicircum \textbackslash
    \item Intention: Figure out how to make table of contents link to the relevant section within the document.
    \item Action: Searched on google for solution.
    \item Outcome: Could not find what I was after. Referred to classmates TeX code and copied it to my own document. Success with making the table of contents link to sections within itself.
\end{itemize}

10 NOV 8:32 PM

\begin{itemize}
    \item Intention: Download Overleaf project from Terminal with git clone.
    \item Action: Over the past few days I was unsuccessfully attempting to download the git clone of my Overleaf project. I was trying to download the project using the `curl -O [URL]' command I found online. This command was successfully working for downloading PDF documents so I figured it would also work for the git clone.
    \item Outcome: After fiddling with different variations of commands on and off over the past few days it clicked that I did not need to use the curl command at all! Copying the entire line supplied in the Overleaf Menu `git clone https://git.overleaf.com/[projectnumber]' led to Terminal prompting me to type my username for Overleaf, then it prompted my password. After typing them in it allowed me to successfully download the project to the directory of my choosing. I had previously downloaded Git from \href{https://git-scm.com/downloads}{this website} so there was no error in using the command for git.
    \item Intention: To successfully convert the downloaded TeX file to Docx with Pandoc.
    \item Actions: Found a page on the \href{https://pandoc.org/demos.html}{Pandoc website} with example scripts for Terminal. Example number 30 was the TeX to Docx Shell command `pandoc -s example.tex -o example.docx'. I opened up a Terminal and tested the command on a previously downloaded TeX file.
    \item Outcome: Test of command was successful. I opened the new Docx file and it was a successful conversion.
    \item Intention: Read about a software called `Rakali'.
    \item Action: On this \href{https://blog.martinfenner.org/2014/08/18/introducing-rakali/}{blog post} by Martin Fenner on 18 August 2014 he describes what Rakali is. He writes that it is a ``Ruby gem that acts as a wrapper for the Pandoc universal document converter...built...to make it easier to use Pandoc to convert large numbers of documents in an automated way:
    \begin{itemize}
        \item bulk conversion of all files in a folder with a specific extension, e.g. md.
        \item input via a configuration file in yaml format instead of via the command line
        \item validation of documents via JSON Schema, using the json-schema Ruby gem.
        \item Logging via stdout and stderr."
    \end{itemize}
    \item Outcome: This looks like a very interesting and useful piece of software however time restraints lean me towards not using it for the time being. I may look into it at a later date for my future academic work.
\end{itemize}

\end{document}

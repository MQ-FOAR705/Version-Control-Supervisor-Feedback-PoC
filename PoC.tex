\documentclass{article}
\usepackage{geometry}
\usepackage[utf8]{inputenc}
\usepackage{graphicx}
\graphicspath{ {images/} }
\geometry{a4paper, left=20mm, right=20mm, top=25mm, bottom=25mm}

\title{Version-Control-Supervisor-Feedback-PoC}
\author{Jeremy Amin}
\date{November 2019}

%\pagestyle{plain}
\usepackage{fancyhdr}
\pagestyle{fancy}
\fancyhf{}
\rhead{J. Amin - 42107881}
\lhead{FOAR705 - Supervisor Feedback PoC}
\lfoot{Session 2, 2019 - Macquarie University}
\rfoot{\thepage}

\usepackage[colorlinks=true,linkcolor=blue]{hyperref}
\usepackage{hyperref}

\begin{document}

\maketitle
\tableofcontents

\pagebreak

\section{Introduction}

This document is my new PoC. My plan is to create a workflow which allows me to receive supervisor feedback in a version controlled location, implement said feedback, and then send a report to my supervisor demonstrating that I have implemented their feedback.

\section{Scoping I}

\subsection{A Day in the Life: Feedback from my supervisor on the written component of my thesis}

I need to:
\begin{itemize}
    \item Have the meeting with my supervisor and record their feedback on my progress or have them send feedback in a docx file.
    \item Make sure I record important actionable feedback in such a way that it is easily understood by myself and them at a later date.
    \item Save the feedback in a version controlled format which I will not lose even if my computer crashes.
    \item Implement their feedback and record it precisely so that I can demonstrate that I have met their demands/suggestions.
    \item Send the implemented actions to my supervisor in an easy to use format for them. (Probably Txt, Docx, or PDF)
    \item Just in case there is a contradiction between their initial feedback and my appropriate implementation of said feedback, I will have a copy of what they have suggested ready to hand.
    \item Just in case they wish to see precisely what I have changed, I will have a history of my edits to my thesis so I can show what I changed and how it changed.
\end{itemize}

\subsection{\textit {Pains} and pain relievers}

\begin{itemize}
    \item \textit{Pain 1: Manually searching through my document(s) to find the specific alteration(s) I have made.} 
    \begin{itemize}
        \item  Reliever: Having a version controlled system in place which enables me to have a history of my alterations as well as an easy way to compare older versions of what I have written to compare with the updated text based on supervisor feedback. 
    \end{itemize}
    \item \textit{Pain 2: Demonstrating precisely why my alteration is better than the previous version.} 
    \begin{itemize}
        \item  Reliever 1: A set of themed repositories and folders which enable me to search for my feedback based alterations based on: Date of feedback, time and date feedback was implemented, section of thesis, page number.
        \begin{itemize}
        \item Reliever 1.1: Being able to compare the current version of a section of text with its older versions. I would then send the relevant versions to my supervisor so they can see the precise alterations I made and see how the alterations satisfy their feedback.
        \end{itemize}
    \end{itemize}
\end{itemize}

\section{Scoping II: Computational Analysis}


\subsection{Goal}

To create a workflow, preferably with some kind of code, to eliminate at least one manual step in the chain of action from receiving feedback to demonstrating that I have implemented feedback.

\subsection{Decomposition}

The process of receiving feedback from my supervisor may proceed as follows:

\begin{itemize}
    \item Commit Overleaf project to GitHub
    \item Download TeX file from Overleaf onto my computer
    \item Convert TeX file to Docx
    \item Send Docx file to supervisor
    \item Receive commented Docx file from supervisor
    \begin{itemize}
    \item Download the Docx file and open it in Microsoft Office.
    \end{itemize}
    \item Find the sections with comments in the document.
    \item Copy the feedback from the comment boxes to Overleaf.
    \item Make alterations to the Overleaf file.
    \item Once all alterations are done to my satisfaction, commit the changes to GitHub and write a commit message which notes the changes made.
    
\end{itemize}

\subsection{Pattern Recognition}

The patterns I am going to focus on are:

\begin{itemize}
    \item Copying the in body text and comments from the Docx file to txt or md.
    \item Saving the txt or md files with a version history.

\end{itemize}

\subsection{Algorithm Design}

\begin{itemize}
    \item Use terminal to quickly interact with Pandoc, Word, Git, and GitHub.
\end{itemize}

\section{Elaboration I}


\noindent
I notice that I did not make the Algorithm part of Scoping II detailed enough. It seems I misunderstood the difference between it and the Decomposition section. In the Elaboration process section I rectify this mistake.


\subsection{Potential Software to solve my problems}

\subsubsection{Terminal on OSx}

Downloading TeX script from Overleaf git clone.

\begin{itemize}
    \item Command to download a file from the internet: git clone
\end{itemize}

\subsubsection{OverLeaf}

Document writing with more control over the formatting etc. compared to Word or Google Documents. Also enables version control in a seamless way.

\subsubsection{Git}

A program which allows for local version controlling.

\subsubsection{GitHub}

This will be where I have the online version controlled repository for the feedback.

\subsubsection{Pandoc}

Pandoc will allow me to convert from TeX to Docx. I may need to convert TeX to Markdown, then Markdown to Docx

\subsubsection{pdf2docx.com/}

This is a free website which converts PDF's to Docx

\section{Elaboration II}

Note: Before updating my macOS, programs such as Microsoft Word and Pandoc would not run as far as I could tell. Updating to the latest macOS (Catalina Version 10.15.1) fixed this problem. Prior to this I was using OS X El Capitan and was running into incompatibility issues frequently enough to warrant an update.

\subsection{Pandoc}

Pandoc is an easy to use program. Once it is installed, a few commands in the macOS terminal allow for quick conversions from and to multiple file types. I tested a simple procedure to create and then convert a Docx file to txt and md files respectively. The procedure was a success.

 \begin{itemize}
        \item In the folder of my choice I opened a terminal by right clicking on the folder and selecting `New Terminal at Folder'.
        \item Once in terminal, I input `touch file.docx' to create a new Docx file.
        \item Then I opened up the new file and wrote a few sentences in the body of the document and also wrote a few comments for the sentences in the body of the document.
        \item Then back in terminal I ran the following two commands.
        \begin{itemize}
            \item pandoc -s file.docx -o file.txt
            \item pandoc -s file.docx -o file.txt
        \end{itemize}
        \item These commands created a txt and md file respectively but without the comments I wrote in Word.
        \item Modifying the pandoc command to include an argument which would retain the changes made to the Docx file produced a different output.
        \begin{itemize}
            \item pandoc --track-changes=all file.docx -o file.txt
        \end{itemize}
        \item The inclusion of --track-changes=all successfully generated a txt file which retained the comments I wrote in the Docx file.
\end{itemize}

The next step in my testing phase was to figure out how to create a local repository with version controlling abilities.

\subsection{Git \& GitHub}

For Git and GitHub \href{https://www.youtube.com/watch?v=SWYqp7iY_Tc}{this YouTube video by Traversy Media} was quite helpful. The instructor simply explained a few commands for Git and GitHub from the terminal command line.

\subsubsection{Git}

Using Git was initially confusing as I was not able to open a GUI for it. After fiddling around with it for a while it clicked that I could just use it via terminal. This proved more efficient and less prone to error than relying on the GUI side of the program. My test included:

\begin{itemize}
    \item Creating a local Git repository. In the same terminal which was in the directory of my files which were converted using Pandoc I wrote in the command line:
    \begin{itemize}
        \item git init
    \end{itemize}
    \item This initialized an empty Git repository in that directory.
    \item I then wrote a few more commands by following the instructions from the aforementioned video to add and commit my preferred files to the Git repository. A few commands involved include:
    \begin{itemize}
        \item git add - This selected files to be added to the Git repository.
        \item git commit - This committed the selected files to the Git repository.
    \end{itemize}
\end{itemize}

\subsubsection{GitHub}

GitHub was used by us in class from the beginning of semester for our version control. I decided it would be useable for my PoC purposes as an online and easy to use interface for comparing document and comment changes from a Docx file.

\begin{itemize}
    \item This command connected my local repository with one on GitHub I had reviously made:
    \begin{itemize}
        \item git remote add origin [URL]
    \end{itemize}
    \item This command pushed the content of my local Git repository to the repository on GitHub I created:
    \begin{itemize}
        \item git push -u origin master
    \end{itemize}
    \item To push any further changes done in my local Git repository to GitHub I used this command in terminal:
    \begin{itemize}
        \item git push
    \end{itemize}
    \item And this command to pull changes done in the repository on GitHub:
    \begin{itemize}
        \item git pull
    \end{itemize}
    
\end{itemize}

\section{Proof of Concept Design}

\subsection{User stories}

\subsubsection{Comments in Word}

\begin{itemize}
    \item As a researcher I want to be able to send and receive feedback in docx file format and version control the comments from my supervisor.
\end{itemize}

\subsubsection{Acceptance Criteria}

As a research student I should be able to:

\begin{enumerate}
    \item Receive a docx document which contains feedback on my work from my supervisor.
    \item Convert the docx document to a non-binary file format (eg txt or md) which I can easily manipulate on terminal, Git and GitHub.
    \item Synchronise both the Docx file and the non-binary file to a local Git repository.
    \item Synchronise the local repository to a GitHub repository.
\end{enumerate}

\section{Implementation}

I have chosen to focus mostly on the docx to txt conversion via Panoc and version controlling it on GitHub. The implementation of this PoC can be found \href{https://github.com/MQ-FOAR705/Version-Control-Supervisor-Feedback-PoC}{in this repository on GitHub}

\subsection{Using Pandoc in terminal}

\begin{itemize}
    \item Using Pandoc I was able to create files from docx to txt and vice versa while retaining the comments from the docx file.
    \item I was also able to use Pandoc and macOS text editor in such a way that I could modify a txt file and then convert it to docx format and create comments in this fashion. \href{https://github.com/MQ-FOAR705/Version-Control-Supervisor-Feedback-PoC/tree/master/aa}{This directory} in the repository contains files which I created and manipulated in just this manner. Read the file \href{https://github.com/MQ-FOAR705/Version-Control-Supervisor-Feedback-PoC/blob/master/aa/Contents.md}{Contents.md} for an example of what the process would look like. The file contains four screenshots. The first is an initial docx file, the two in the middle are txt files which show the edits I make between, and the final screenshot is the output docx file with the additions to the original content.
\end{itemize}

\subsection{Using Git \& GitHub}

I have written a document which details the steps I took in using Git and GitHub with Pandoc and terminal on macOS. \href{https://github.com/MQ-FOAR705/Version-Control-Supervisor-Feedback-PoC/blob/master/Step-guide.md}{This} link will take you to the document. Section 4 and 5 in that document has instructions for using Git and GitHub with terminal and in GUI.

\section{Disaster recovery plan}

\begin{itemize}
    \item With the exception of Microsoft Word, all of the programs involved in this PoC are all open source. All of the programs involved in this PoC are downloadable should any one or all of them malfunction.
    \item The repository which contains the information for this PoC is stored in five locations. Two are local: On my laptop, and on an external hard disk back up. The other three locations are online: On the GitHub website in a \href{https://github.com/MQ-FOAR705/Version-Control-Supervisor-Feedback-PoC}{repository}, in an account on the Overleaf website, and in an account on Cloudstor.
\end{itemize}

\end{document}

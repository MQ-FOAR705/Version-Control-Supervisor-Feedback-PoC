\documentclass{article}
\usepackage{geometry}
\usepackage[utf8]{inputenc}
\usepackage{graphicx}
\graphicspath{ {images/} }
\geometry{a4paper, left=20mm, right=20mm, top=25mm, bottom=25mm}

\title{Version-Control-Supervisor-Feedback-PoC}
\author{Jeremy Amin}
\date{November 2019}

%\pagestyle{plain}
\usepackage{fancyhdr}
\pagestyle{fancy}
\fancyhf{}
\rhead{J. Amin - 42107881}
\lhead{FOAR705 - Supervisor Feedback PoC}
\lfoot{Session 2, 2019 - Macquarie University}
\rfoot{\thepage}

\usepackage[colorlinks=true,linkcolor=blue]{hyperref}
\usepackage{hyperref}

\begin{document}

\maketitle
\tableofcontents

\pagebreak

\section{Introduction}

This document is my new PoC. My plan is to create a workflow which allows me to receive supervisor feedback in a version controlled location, implement said feedback, and then send a report to my supervisor demonstrating that I have implemented their feedback.

\section{Scoping I}

\subsection{A Day in the Life: Feedback from my supervisor on the written component of my thesis}

I need to:
\begin{itemize}
    \item Have the meeting with my supervisor and record their feedback on my progress or have them send feedback in a Docx file.
    \item Make sure I record important actionable feedback in such a way that it is easily understood by myself and them at a later date.
    \item Save the feedback in a version controlled format which I will not lose even if my computer crashes.
    \item Implement their feedback and record it precisely so that I can demonstrate that I have met their demands/suggestions.
    \item Send the implemented actions to my supervisor in an easy to use format for them. (Probably Txt, Docx, or PDF)
    \item Just in case there is a contradiction between their initial feedback and my appropriate implementation of said feedback, I will have a copy of what they have suggested ready to hand.
    \item Just in case they wish to see precisely what I have changed, I will have a history of my edits to my thesis so I can show what I changed and how it changed.
\end{itemize}

\subsection{\textit {Pains} and pain relievers}

\begin{itemize}
    \item \textit{Pain 1: Manually searching through my document(s) to find the specific alteration(s) I have made.} 
    \begin{itemize}
        \item  Reliever: Having a version controlled system in place which enables me to have a history of my alterations as well as an easy way to compare older versions of what I have written to compare with the updated text based on supervisor feedback. 
    \end{itemize}
    \item \textit{Pain 2: Demonstrating precisely why my alteration is better than the previous version.} 
    \begin{itemize}
        \item  Reliever 1: A set of themed repositories and folders which enable me to search for my feedback based alterations based on: Date of feedback, time and date feedback was implemented, section of thesis, page number.
        \begin{itemize}
        \item Reliever 1.1: Being able to compare the current version of a section of text with its older versions. I would then send the relevant versions to my supervisor so they can see the precise alterations I made and see how the alterations satisfy their feedback.
        \end{itemize}
    \end{itemize}
\end{itemize}

\section{Scoping II: Computational Analysis}


\subsection{Goal}

To create a workflow, preferably with some kind of code, to eliminate at least one manual step in the chain of action from receiving feedback to demonstrating that I have implemented feedback.

\subsection{Decomposition}

The process of receiving feedback from my supervisor may proceed as follows:

\begin{itemize}
    \item Commit Overleaf project to GitHub
    \item Download TeX file from Overleaf onto my computer
    \item Convert TeX file to Docx
    \item Send Docx file to supervisor
    \item Receive commented Docx file from supervisor
    \begin{itemize}
    \item Download the Docx file and open it in Microsoft Office.
    \end{itemize}
    \item Find the sections with comments in the document.
    \item Copy the feedback from the comment boxes to Overleaf.
    \item Make alterations to the Overleaf file.
    \item Once all alterations are done to my satisfaction, commit the changes to GitHub and write a commit message which notes the changes made.
    
\end{itemize}

\subsection{Pattern Recognition}

The patterns I am going to focus on are:

\begin{itemize}
    \item Copying the in body text and comments from the Docx file to txt or md.
    \item Saving the txt or md files with a version history.

\end{itemize}

\subsection{Algorithm Design}

\begin{itemize}
    \item Use terminal to quickly interact with Pandoc, Word, Git, and GitHub.
\end{itemize}

\section{Elaboration I}


\noindent
I notice that I did not make the Algorithm part of Scoping II detailed enough. It seems I misunderstood the difference between it and the Decomposition section. In the Elaboration process section I rectify this mistake.


\subsection{Potential Software to solve my problems}

\subsubsection{Terminal on OSx}

Downloading TeX script from Overleaf git clone.

\begin{itemize}
    \item Command to download a file from the internet: git clone
\end{itemize}

\subsubsection{OverLeaf}

Document writing with more control over the formatting etc. compared to Word or Google Documents. Also enables version control in a seamless way.

\subsubsection{Git}

A program which allows for local version controlling.

\subsubsection{GitHub}

This will be where I have the online version controlled repository for the feedback.

\subsubsection{Pandoc}

Pandoc will allow me to convert from TeX to Docx. I may need to convert TeX to Markdown, then Markdown to Docx

\subsubsection{pdf2docx.com/}

This is a free website which converts PDF's to Docx

\section{Elaboration II}

Note: Before updating my macOS, programs such as Microsoft Word and Pandoc would not run as far as I could tell. Updating to the latest macOS (Catalina Version 10.15.1) fixed this problem. Prior to this I was using OS X El Capitan and was running into incompatibility issues frequently enough to warrant an update.

\subsection{Pandoc}

Pandoc is an easy to use program. Once it is installed, a few commands in the macOS terminal allow for quick conversions from and to multiple file types. I tested a simple procedure to create and then convert a Docx file to txt and md files respectively. The procedure was a success.

 \begin{itemize}
        \item In the folder of my choice I opened a terminal by right clicking on the folder and selecting `New Terminal at Folder'.
        \item Once in terminal, I input `touch file.docx' to create a new Docx file.
        \item Then I opened up the new file and wrote a few sentences in the body of the document and also wrote a few comments for the sentences in the body of the document.
        \item Then back in terminal I ran the following two commands.
        \begin{itemize}
            \item pandoc -s file.docx -o file.txt
            \item pandoc -s file.docx -o file.txt
        \end{itemize}
        \item These commands created a txt and md file respectively but without the comments I wrote in Word.
        \item Modifying the pandoc command to include an argument which would retain the changes made to the Docx file produced a different output.
        \begin{itemize}
            \item pandoc --track-changes=all file.docx -o file.txt
        \end{itemize}
        \item The inclusion of --track-changes=all successfully generated a txt file which retained the comments I wrote in the Docx file.
\end{itemize}

The next step in my testing phase was to figure out how to create a local repository with version controlling abilities.

\subsection{Git \& GitHub}

For Git and GitHub \href{https://www.youtube.com/watch?v=SWYqp7iY_Tc}{this YouTube video by Traversy Media} was quite helpful. The instructor simply explained a few commands for Git and GitHub from the terminal command line.

\subsubsection{Git}

Using Git was initially confusing as I was not able to open a GUI for it. After fiddling around with it for a while it clicked that I could just use it via terminal. This proved more efficient and less prone to error than relying on the GUI side of the program. My test included:

\begin{itemize}
    \item Creating a local Git repository. In the same terminal which was in the directory of my files which were converted using Pandoc I wrote in the command line:
    \begin{itemize}
        \item git init
    \end{itemize}
    \item This initialized an empty Git repository in that directory.
    \item I then wrote a few more commands by following the instructions from the aforementioned video to add and commit my preferred files to the Git repository. A few commands involved include:
    \begin{itemize}
        \item git add - This selected files to be added to the Git repository.
        \item git commit - This committed the selected files to the Git repository.
    \end{itemize}
\end{itemize}

\subsubsection{GitHub}

GitHub was used by us in class from the beginning of semester for our version control. I decided it would be useable for my PoC purposes as an online and easy to use interface for comparing document and comment changes from a Docx file.

\begin{itemize}
    \item This command connected my local repository with one on GitHub I had reviously made:
    \begin{itemize}
        \item git remote add origin [URL]
    \end{itemize}
    \item This command pushed the content of my local Git repository to the repository on GitHub I created:
    \begin{itemize}
        \item git push -u origin master
    \end{itemize}
    \item To push any further changes done in my local Git repository to GitHub I used this command in terminal:
    \begin{itemize}
        \item git push
    \end{itemize}
    \item And this command to pull changes done in the repository on GitHub:
    \begin{itemize}
        \item git pull
    \end{itemize}
    
\end{itemize}

\section{Proof of Concept Design}

\subsection{User stories}

\subsubsection{Comments in Word}

\begin{itemize}
    \item As a researcher I want to be able to send and receive feedback in Docx file format and verion control the comments from my supervisor.
\end{itemize}

\subsubsection{Acceptance Criteria}

As a research student I should be able to:

\begin{enumerate}
    \item Receive a Docx document which contains feedback on my work from my supervisor.
    \item Convert the Docx document to a non-binary file format (eg txt or md) which I can easily manipulate on terminal, Git and GitHub.
    \item Synchronise both the Docx file and the non-binary file to a local Git repository.
    \item Synchronise the local repository to a GitHub repository which will 
\end{enumerate}

\section{Learning Journal for new PoC}

\subsection{7 NOV}

1:30 PM
\begin{itemize}
    \item Intention: Figure out how to convert TeX document to Docx.
    \item Action: Spoke with Brian in a consultation. He send me this article to read: https://medium.com/@zhelin\-chen91/how-to-convert-from-latex-to-ms-word-with-pandoc-f2045a762293
    \item Outcome: It seems that Pandoc is a very good candidate software for converting TeX into Docx. Pandoc is both free and open source, two positives for a piece of software I am intending on using in a workflow for multiple users.
\end{itemize}

1:46 PM
\begin{itemize}
    \item Intention: Download Pandoc for my Macbook
    \item Action: Went to: https://pandoc.org/installing.html to find the installer for Mac. Clicked on the button at the top of the page `Download the latest installer for macOS.' Followed instructions and installed Pandoc. 
    \item Outcome: It seems that I need to have either an older version of Pandoc, or I need to install a newer macOS. On the Homebrew Formulae website it states in the `bottle' section `mojave, high\textunderscore sierra, sierra'. So it seems the current version of Pandoc is only compatible with those three macOS versions.
    \item Solution: I will look for an older version of Pandoc compatible with OS X El Capitan Version 10.11.6. If I cannot find a compatible version of Pandoc I may have to update my OS. I have found a number of programs do not run on my OS. Considering my OS is a number of years old I am not surprised.
\end{itemize}

2:05 PM

\begin{itemize}
    \item Intention: Find a version of Pandoc which works on my Macbook
    \item Action: Went to: https://stackoverflow.com/questions/48430440/install-older-version-of-pandoc-2-using-homebrew from a google search for compatible versions of Pandoc with macOS El Capitan 10.11.6.
    \item Outcome: The answer provided three solutions. One option was to use old formulas. The second was to use old bottles. The third was to ``Build your own retro-formula". I decided to try the second option.
    \item Following the instructions I went to this website: https://bintray.com/homebrew/bottles/pandoc/1.16.0.\-2\#files which had this file https://bintray.com/homebrew/bottles/download\textunderscore  file?file\textunderscore path=pandoc-1.16.0\-.2.el\textunderscore capitan.bottle.tar.gz
    \item Solution: I will look for an older version of Pandoc compatible with OS X El Capitan Version 10.11.6. If I cannot find a compatible version of Pandoc I may have to update my OS. I have found a number of programs do not run on my OS. Considering my OS is a number of years old I am not surprised.
\end{itemize}

2:35 PM

\begin{itemize}
    \item Intention: Figure out how to remove hbox warning in Overleaf when I paste website URL's into the document.
    \item Action: Went to TeX Stack Exchange to find solution.
    \item Outcome: Found solution. I decided to hypenate where appropriate to make the URL fit into the document.

    \item Intention: Figure out how to add \# character into document.
    \item Action: Went to TeX Stack Exchange to find solution.
    \item Outcome: Found solution. Adding {\textbackslash} before the \# solved my problem.
    \item Note: Each of these characters have special meaning in TeX. \& \% \$ \# \_ \{ \} \textasciitilde \textasciicircum \textbackslash
    \item Intention: Figure out how to make table of contents link to the relevant section within the document.
    \item Action: Searched on google for solution.
    \item Outcome: Could not find what I was after. Referred to classmates TeX code and copied it to my own document. Success with making the table of contents link to sections within itself.
\end{itemize}

10 NOV 8:32 PM

\begin{itemize}
    \item Intention: Download Overleaf project from Terminal with git clone.
    \item Action: Over the past few days I was unsuccessfully attempting to download the git clone of my Overleaf project. I was trying to download the project using the `curl -O [URL]' command I found online. This command was successfully working for downloading PDF documents so I figured it would also work for the git clone.
    \item Outcome: After fiddling with different variations of commands on and off over the past few days it clicked that I did not need to use the curl command at all! Copying the entire line supplied in the Overleaf Menu `git clone https://git.overleaf.com/[projectnumber]' led to Terminal prompting me to type my username for Overleaf, then it prompted my password. After typing them in it allowed me to successfully download the project to the directory of my choosing. I had previously downloaded Git from \href{https://git-scm.com/downloads}{this website} so there was no error in using the command for git.
    \item Intention: To successfully convert the downloaded TeX file to Docx with Pandoc.
    \item Actions: Found a page on the \href{https://pandoc.org/demos.html}{Pandoc website} with example scripts for Terminal. Example number 30 was the TeX to Docx Shell command `pandoc -s example.tex -o example.docx'. I opened up a Terminal and tested the command on a previously downloaded TeX file.
    \item Outcome: Test of command was successful. I opened the new Docx file and it was a successful conversion.
    \item Intention: Read about a software called `Rakali'.
    \item Action: On this \href{https://blog.martinfenner.org/2014/08/18/introducing-rakali/}{blog post} by Martin Fenner on 18 August 2014 he describes what Rakali is. He writes that it is a ``Ruby gem that acts as a wrapper for the Pandoc universal document converter...built...to make it easier to use Pandoc to convert large numbers of documents in an automated way:
    \begin{itemize}
        \item bulk conversion of all files in a folder with a specific extension, e.g. md.
        \item input via a configuration file in yaml format instead of via the command line
        \item validation of documents via JSON Schema, using the json-schema Ruby gem.
        \item Logging via stdout and stderr."
    \end{itemize}
    \item Outcome: This looks like a very interesting and useful piece of software however time restraints lean me towards not using it for the time being. I may look into it at a later date for my future academic work.
\end{itemize}

\section{Before we Start - 30 OCT 2019}

Intention: To complete lesson 1 of R for Social Scientists

Actions:

\subsection{What is R? What is RSTudio?}
\begin{itemize}
    \item `R' refers to both the programming language as well as the software that interprets the scripts written using it.
    \item RStudio is a popular current way to write R scrips and interact with R software.
    
\subsection{Why learn R?}

    \item R is preferable for a number of reasons. 1. It relies on written commands, not a mouse. (Written commands may not be as user friendly for the typical GUI user but replicability is far more reliable. 2. Working with scripts forces the user to be more intimately connected with their analysis or whatever other task they wish to conduct with R. This facilitates learning and grants more control to the user. 3. R is interdisciplinary and extensible. It allows for the application of more than one discipline's methods of analysis for data analysis. 4. R allows for the production of high-quality graphics. 5. R has a community of users and people with know-how in case a person needs help. 6. R is free, open-source and cross platform.
    \item Opened a New project and created a new file within the project.
    \item Default layout for RStudio is: 
    \begin{itemize}
        \item Top Left - Source: pane for my scripts and documents
        \item Bottom Left - Console: pane to show me what R would look and be like without RStudio
        \item Top Right - Enviornment/History: pane to show me what I have done.
        \item Bottom Right - Files and more: pane to allow me to see the contents of the project/working directory
    \end{itemize}
\item Created folders from Console using commands: dir.create("data"), dir.create("data\textunderscore output"), and dir.create

\subsection{Exercise: Use the install option from the packages tab to install the ‘tidyverse’ package.}
\item I installed the `tidyverse' package from the drop-down menu. If I want to download packages in the future I can also type the following command into the Console: install.packages()

\end{itemize}

\section{Introduction to R - 4 NOV 2019}
Intention: To complete the Introduction to R lesson.

Actions:

\subsection{Exercise: What do you think is the current content of the object area\textunderscore acres? 123.5 or 6.175?}
\begin{itemize}
    \item The correct answer is 6.175 because although the value of area\textunderscore hectares was changed to 50 this is written after the line to multiply it by 2.47 to convert it into acres.
    \item I assigned values for length, width, and area. Below are the lines I used.
    \begin{itemize}
    \item[] \textgreater length \textless- 5
    \item[] \textgreater width \textless- 10
    \item[] \textgreater area \textless- length * width
    \item[] \textgreater area
    \item[] [1] 50
    \end{itemize}
\item The value of area did not change when I changed the values of width and length. This is for the same reason as the previous exercise. Once I reassigned area to be length multiplied by width the new values of length and width were used and a new value of area was made.
\item Functions are ready made scripts. 
\item Typing ?round into the console brought up the Help pane in the bottom right pane of RStudio. Other functions similar to round are ceiling, floor, trunc, and signif.
\item ceiling and floor seem to perform opposite functions. Ceiling takes whatever argument x and returns the rounded up integer. EG ceiling(1) outputs 1. ceiling(1.0001) outputs 2. floor does something like the opposite. floor(1) outputs 1. floor(1.9) outputs 1 as well.
\item trunc seems to round the input down to the nearest integer to zero. trunc(1.1) outputs 1. trunc(.9) outputs 0. signif  outputs the rounded value in its first argument by the specified number of significant digits in the second argument. This function seems to only output up to 6 significant digits. signif(1.123456789, 7) outputs [1] 1.123467. signif(1.123456789, 8) also outputs [1] 1.123457.
\item For the round function, the digits argument indicates the number of decimal places to be used. It is used by placing a comma after the first argument. E.G. round(x, digits)
\end{itemize}

Continuing 5 NOV 2019

\begin{itemize}
    \item When mixing vector types into a single vector, R implicitly converts them into a single type. When I made a vector with three numeric types and one character type, the typeof function output the type as character. This means R converted the three numeric arguments into character types.
    \item 
\end{itemize}

\end{document}
